%% LyX 2.0.8.1 created this file.  For more info, see http://www.lyx.org/.
%% Do not edit unless you really know what you are doing.
\documentclass[english]{article}
\usepackage[latin9]{inputenc}
\usepackage{listings}
\usepackage{babel}
\begin{document}

\title{OpenCL in R by example}


\author{Marek Kudla}


\date{12/04/15}

\maketitle

\section*{Introduction}

For laymen: OpenCL is a High-Throughput Computing language that enables
use of computation accelerators, like GPUs, multicore CPUs or specialized
cards. R is a statistical programming language that provides many
built-in and verified statistical functions as well as rich plotting
capabilities, yet until recently was single-threaded, so ill-suited
for running computer-intensive simulations. With the advent of OpenCL
it is now possible to bring supercomputer-scale simulations to inexpensive
workstations. Setting the OpenCL in R has many advantages. You can
use computing power of OpenCL for number crunching, while still process
input and output with high-level R functions. Typical example would
be generating random numbers in R using its various distribution functions,
run intensive calculations on them using OpenCL, then use R to plot
the results. Such set-up makes it possible for you to skip solving
non-essential programming problems in C or C++ and focus on writing
computational code.


\section*{Using Multiple Cores of Modern Microprocessors in R}

Generally R code runs on a single core. It does not automatically
use all available cores. However it is possible to convince it if
you explicitely ask for it nicely. If your input is a list of mutually
independent values then you can divide this list and run multiple
copies of your function, each utilising free core. This is available
through 'parallel' package that is now built-in into the R itself.
Typical computers have 1-8 cores, computational servers can have up-to
64 cores. Speed-up with this approach is therefore limited, but messing
with OpenCL is not required.

You need to put all your code in the function and then invoke it within
the mclapply function - a parallelized version of lapply. Its mc.cores
option specifies how many cores you want to use. There are other functions
in this package: mcmapply and mcmap providing similar functionality.

\begin{lstlisting}[basicstyle={\small\ttfamily},breaklines=true,language=R]
library(parallel)
k<-1:100000                      # create a list of inputs  
par2x<-function(x)   
	return(2*x)                  # define function
mclapply(k, par2x, mc.cores=4)   # run multiple copies of function
\end{lstlisting}


It is easy to get the number of cores available:

\begin{lstlisting}[basicstyle={\small\ttfamily},language=R]
detectCores() [1] 4
\end{lstlisting}



\section*{OpenCL Installation on R}

I recommend using a 64-bit distribution of R, along with the RStudio,
since it is a great IDE and saves time overall. On Linux, Windows
you need to install OpenCL driver for the GPU you are using. Sometimes
things get dirty and in the appendix I have described an example of
a more complex scenario. On mac, the system OpenCL libraries are already
installed and the only thing left is to install OpenCL package for
R by Simon Urbanek.

\begin{lstlisting}[basicstyle={\small\ttfamily},breaklines=true,language=R]
>install.packages("OpenCL")
Installing package into '/Users/marekkudla/Library/R/3.1/library' (as 'lib' is unspecified)

package 'OpenCL' is available as a source package but not as a binary

Warning in install.packages :   package 'OpenCL' is not available (for R version 3.1.1) 
\end{lstlisting}


Ooops, there is no binary form available, but we can compile our own:

\begin{lstlisting}[basicstyle={\small\ttfamily},language=R]
> install.packages("OpenCL", type="source")
\end{lstlisting}


then let\textquoteright{}s load the library:

\begin{lstlisting}[basicstyle={\small\ttfamily},language=R]
> library(OpenCL)
\end{lstlisting}


and check if it works:

\begin{lstlisting}[basicstyle={\small\ttfamily}]
> p=oclPlatforms()
> p
[[1]] OpenCL platform 'Apple'
> d=oclDevices(p[[1]]) 
> d
[[1]] OpenCL device 'Iris Pro' 
\end{lstlisting}


Ok, this is cool, because the Iris Pro in this case is 5200 of 4th
gen Core processor, a modest GPU that should give about 700 GFLOPS/s.
Currently the top of the line GPUs can give 8000 GFLOPS/s, while processors
give approximately 100 GFLOPS/s, but only if accessed through OpenCL
(usually less).


\section*{Simple Kernel and Program Skeleton}

Rememeber that, even though you are in R environment, you still need
to program in OpenCL. You need to be familiar with C, but that is
similar enough to R to learn quickly. The language is basically limited
C99 with additional built-in math functions. You can use it to process
various integers, floating points and even char (bitfield) types.
The Simonek's OpenCL implementation has some limitations though: most
important is that it is limited to floating point numbers, so strings
are out of question. The other one is that while you can use multiple
inputs, only one output. Those limitations could be worked around,
but would require major effort, so perhaps it is better to move the
tasks that would require it to C/C++ host programs. In this way, the
R/OpenCL remains a platform for fast prototyping for floating point
numbers. Let's create a simple example of calculations in OpenCL on
R. And let's be a bit more ambitious and create a good starting point
for further modifications. The example program takes two tables of
floats as input (same size), two single float variables and outputs
a table of floats.

\begin{lstlisting}[basicstyle={\footnotesize\ttfamily},breaklines=true]
library(OpenCL) 
p=oclPlatforms() 
d=oclDevices(p[[1]])
# the library is loaded 
# we create platform p
# and device d

#
# OpenCL kernel
#
code1 = c("
__kernel void kern1(
	__global float* output,
	const unsigned int count,
	__global float* input1,
	__global float* input2,
	const float alpha,
	const float beta
)
	{
			/* only C-style comments here */
			// or C++ style comments

		int i = get_global_id(0);	
            // this is the id for your datum
		output[i] = (input1[i] + input2[i]) * alpha + beta;
            // simple calculation
	};
")
#
# end of the kernel code
#

# compiling kernel for specific platform and device
examplekernel1 <- oclSimpleKernel(d[[1]], "kern1", code1, "single")

# embedding kernel in the function
oclf <- function(x, y, alpha = 1.0, beta=0.00 ...) {
	oclRun(examplekernel1, length(x), x, y, alpha, beta, ...)
}
# use oclf as a function on x and y
x<-clFloat(rnorm(1000000, mean=0, sd=1)) # normal distribution
y<-clFloat(rexp(1000000,rate = 0.5)) # exponential distribution
z<-oclf(x,y)

# see the results
hist(as.numeric(x), breaks=60)
hist(as.numeric(y), breaks=60)
hist(z, breaks=60)
\end{lstlisting}


Errors and error checking for inputs is best done outside the OpenCL.
The error messages from kernels are cryptic, so it is best to keep
it simple and have tests to verify its correctness.


\section*{Timing Execution}

Timing execution of the program is simple using R built-in function
system.time(). Just use your function and embed it into the timing
function. Do not use timing over the graphing part of your program,
since then you will measure the delay in the R's graphics pipelinewhich
is usually huge. 

\begin{lstlisting}[basicstyle={\small\ttfamily},language=R]
system.time(g(x))
\end{lstlisting}



\section*{Mandelbrot Fractal}

Let's do something fun. Here is the OpenCL code for creating a Mandelbrot
fractal.

\begin{lstlisting}[basicstyle={\small\ttfamily},breaklines=true]
#############################
## Mandelbrot OpenCL Code 
##############################

## setup OpenCL 
library(OpenCL) 
p=oclPlatforms() 
d=oclDevices(p[[1]])

## kernel code (calculate mandelbrot) 
code2 = c(" 
__kernel void calc1( 
__global float* output, 
const unsigned int count, 
__global float* inputx, 
__global float* inputy) 
{ 
	int i = get_global_id(0); 
	if(i < count){ 
		__private float tempx; 
		__private float tempy; 
		__private float temp; 
		tempx = inputx[i]; 
		tempy = inputy[i]; 
		temp = tempx; 
		for(__private int iter=1; iter<=100; iter++) 
		{ 
			temp = tempx; 
			tempx = tempx * tempx - tempy * tempy + inputx[i]; 
			tempy = 2 * temp * tempy + inputy[i]; 
		} 
		output[i] = tempx*tempx+tempy*tempy; 
	} 
};")


## compile kernel and embeed it into function 
k.calc2 <- oclSimpleKernel(d[[1]], "calc1", code2, "single") 
f <- function(x, y, ...) 
	oclRun(k.calc2, length(x), x, y)

## setup environment
size=4000 
iterations=100 
matxR<-matrix(rep(seq(-2,2,length.out=size),size),size) 
matyR<-t(matxR)
## use function 
image(matrix(f(matxR,matyR),size),zlim=c(-4,4))
## benchmark function 
system.time(f(matxR,matyR))
\end{lstlisting}


If you want to compare execution time, you can run the conventional
version of the program.

\begin{lstlisting}[basicstyle={\small\ttfamily},breaklines=true,language=R]
############################# 
## Mandelbrot Conventional Code 
##############################
## setup environment 
size=4000 
iterations=10 
matxR<-matrix(rep(seq(-2.5,1.5,length.out=size),size),size) 
matyR<-t(matxR)
g <-function(matxR, matyR) 
{ 
	iterx<-matxR 
	itery<-matyR 
	for(i in 1:iterations)  {   
temp<-iterx   
iterx<-iterx*iterx-itery*itery+matxR   
itery<-2*temp*itery+matyR   
} 
return(iterx*iterx+itery*itery) 
}
## use function 
image(matrix(g(matxR,matyR),size),zlim=c(-4,4))
## benchmark function 
system.time(g(matxR,matyR))
\end{lstlisting}


The difference in execution time is very noticable, being 178x of
the OpenCl version.

\rule[0.5ex]{0.9\textwidth}{0.5pt}


\section*{Appendix}

Installation of the Simon Urbanek's OpenCL package in R on 64-bit
linux and AMD GPU. Installation of graphics driver is complicated,
but at least documented and I was able to go through it after few
attempts. The further problem was that Simon's package assumes links
to the libraries are on pathway typical for 32 bit systems and they
look differently on 64 bit ones. Here is how the error looked like:

\begin{lstlisting}[basicstyle={\small\ttfamily},breaklines=true,language=R]
> install.packages("OpenCL") 
Installing package(s) into '/home/user1/R/x86_64-pc-linux-gnu-library/2.14' (as 'lib' is unspecified)
trying URL 'http://watson.nci.nih.gov/cran_mirror/src/contrib/OpenCL_0.1-3.tar.gz' 
Content type 'application/octet-stream' length 15524 bytes (15 Kb) 
opened URL 
================================================== 
downloaded 15 Kb
* installing *source* package 'OpenCL' ... 
** package 'OpenCL' successfully unpacked and MD5 sums checked 
** libs gcc -std=gnu99 -I/usr/share/R/include      -fpic  -O3 -pipe  -g -c ocl.c -o ocl.o 
ocl.c:6:23: fatal error: CL/opencl.h: No such file or directory 
compilation terminated. 
make: *** [ocl.o] Error 1 
ERROR: compilation failed for package 'OpenCL'
* removing '/home/user1/R/x86_64-pc-linux-gnu-library/2.14/OpenCL' 
Warning in install.packages :   installation of package 'OpenCL' had non-zero exit status
The downloaded packages are in 	
'/tmp/RtmprdyHgJ/downloaded_packages' 
\end{lstlisting}


and here is the fix:

\begin{lstlisting}[basicstyle={\small\ttfamily},breaklines=true]
cd /usr/share/R/include 
sudo ln -s /opt/AMDAPP/include/CL ./CL 
cd /usr/lib/R/lib 
sudo ln -s /opt/AMDAPP/lib/x86_64/libOpenCL.so ./libOpenCL.so
\end{lstlisting}


after that it installed normaly and I was able to use it.
\end{document}
